\beginsong{Гимн номер 1}[
  by={Настя Бакушкина, Юля Анисимова},
  sr={СПО <<Со$\oStar$>>},
  cr={2010 г.} ]



{\nolyrics Бой: $\down\q   \down \up\q   \up \down \up $} 
\pchk

\gtab{C}{X32010}
\gtab{G}{320003}
\gtab{Am}{X02210}
\gtab{Dm}{XX0231}
\gtab{E}{022100}

\gtab{A7}{X02020}
\gtab{F}{133211}
\psk


\beginverse
В\[C]се, кто распрощался с детством,
\[G]Нас понять не смогут --- ну и \[Am]пусть.
У них своя дорога,
\[F]Каждый по своему \[G]дышит и рад!
\bqk

А мы нашли одно лишь средство
Как прогнать навек тоску и грусть,
И нужно нам не много:
Просто поехать на смену в отряд!
\endverse


\beginchorus

\[C] Со$\oStar$\dots на \[Am]небосклоне \[A7]детских судеб
В\[Dm]месте мы\dots в их \[F]душах что--ни\[E]будь пробудим
\[C]Ты и я\dotsво\[Am]жатые из \[A7]этой песни
В\[F]месте мы\dots Со$\oStar$ \[G]
\endchorus


\beginverse
Нас беда не сломит никогда,
На жизнь мы выбрали свой путь.
Нам лучшая награда ---
Много отличных и ярких бойцов!
\bqk

Ведь нас не испугают холода,
И глаз своих нам не сомкнуть,
Когда мы всем отрядом
Замыкаем дружбы кольцо!
\endverse


\beginverse
Там, где в строевках зелёных мы,
За стройным рядом кирпичей
Как солнце, светит сердце ---
Счастье для нас быть самими собой!
\bqk

И вдруг, если ты в жизнь влюблённый,
Но пока еще совсем ничей ---
Мы ждём тебя в мир детства,
Зажигай и будь с нами звездой!
\endverse

\endsong


\beginscripture{}
Песня написана в начале лета 2010 года Настей Бакушкиной и Юлей Анисимовой. Слова были переложены на музыку Сашей Алиным из ССО <<Искра>>.
\endscripture

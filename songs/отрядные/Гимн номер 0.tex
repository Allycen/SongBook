\beginsong{Гимн номер 0}[
  by={Даша Портнова},
  sr={СПО <<Со$\oStar$>>}]


{\nolyrics Бой: $\down\q   \down \up\q   \up \down \up $} 
\pchk
	
\gtab{D}{XX0232}
\gtab{A}{X02220}
\gtab{G}{320003}
\gtab{Hm}{2:X13321}
\gtab{F#}{2:133211}
\psk


\beginverse
\[D]А вот \[A]я сейчас с\[D]пою
Стандартную песенку вожатскую.
Про \[G]то, как ночами я не сп\[D]лю,
И вооб\[G]ще, про судь\[A]бину мою \[D]адскую.
\bqk

Официантом я работаю бесплатно,
И уборщицей и воспиталкой тоже.
По утрам крадусь я в душик аккуратно,
Чтобы тёпленьким залыбить свою рожу.
\endverse


\beginchorus
Ниче\[G]го, ничего --- это в\[D]сё ерунда,
Это в\[F#]сё как везде непо\[Hm]нятки.
Но зато, но зато спят уже сладким сном
По кроваткам любимые ребятки.
\bqk

А потом, а потом всем отрядом споём
Мы <<Du Hast>> на утренней зарядке,
Вместо лент, вместо лент всем девчонкам заплетём
Солнца лучиков жёлтенькие прядки.
\endchorus


\beginverse
В туалет я сходить не успеваю,
Кушать сидя теперь для меня счастье,
А директриса, я так подозреваю,
У Саурона увела кольцо всевластия.
\bqk

Мы чихаем под ритмы дискотеки,
И сморкаемся мы все чудесным хором,
И мечтаем побывать в аптеке,
Чтобы голос не звучал хардкором.
\endverse


\beginverse
\dittotikz\ + \dots
И спасибо, спасибо, что Со$\oStar$ сияет
И напарнице спасибо персонально.
Я люблю своих детей, и вообще всех вас люблю!
И заканчиваю песенку банально.
\endverse


\endsong

\beginscripture{}
Песня написана в 2010 году на самой первой смене Со$\oStar$ в лагере <<Факел+>> Дашей Порт\-новой. Она исполнялась на ВП ЛЭТИ в 2011 году в номинации <<Отрядная>>. 
\endscripture



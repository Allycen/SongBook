\beginsong{Ода первому напарнику}[
  txt = {Вика Писарева, Паша Аксёнов},
  musc = {Вика Писарева},
  sr={СПО <<Со$\oStar$>>}]


{\nolyrics \boi $\down\q   \downs\q    \down \up \down \ups$}
\pchk

\gtab{C}{X32010}
\gtab{G}{320003}
\gtab{Dm}{XX0231}
\gtab{Am}{X02210}
\gtab{E}{022100}
\psk



\beginverse
Не\[C]важно на \[G]море ты или з\[Dm]десь рядом,
Ведь детям вожатый нужен всег\[G]да.
Не только призвание, а скорее работа
Дарить малышам частички добра.
\endverse


\beginchorus
\[Dm] Никто не з\[G]нает, что бы \[C]было с \[Am]нами,
\[Dm] Если бы в\[G]месте судь\[C]ба нас \[E]не све\[Am]ла.
Теперь могу я вас назвать друзьями,
Ведь смена наша лучше всех \[E]бы\[Am]ла.
\endchorus


\beginverse
Теперь вспоминаешь первую смену.
Как думал, что будешь всё делать один.
И как познакомился с первым напарником, 
И как провёл три недели с ним.
\bqk

Сначала казался суровым и строгим,
Но быстро ты понял, что это не так.
И он оказался таким ярким и добрым,
Помочь тебе для него --- просто пустяк.
\endverse


\beginverse
И не скрываем --- были и ссоры,
И часто не знали, кто виноват.
Но у нас получилось, всё разрешилось,
И шли мы вместе строить отряд.
\endverse


\beginverse
Порой меня спросят: <<Что же случилось?
Возможно ты где--то свернул не туда?>>
А я отвечу: <<Хорошо получилось,
Ведь Со$\oStar$ вместе всегда!>>
\endverse

\endsong


\beginscripture{}
Первая версия песни написана осенью 2014 года Викой Писаревой, окончательная версия песни была готова весной 2015 года при помо\-щи Паши Аксёнова.
\endscripture

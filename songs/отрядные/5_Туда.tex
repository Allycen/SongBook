\beginsong{Туда}[
  by={Маша Бякина, Вова Чеботарёв},
  sr={СПО <<Со$\oStar$>>}]



{\nolyrics Бой: $\down\q  \down\q  \up \up \down\q  \down \up \down \up$} 
\pchk

\gtab{C}{X32010}
\gtab{Am}{X02210}
\gtab{Em}{022000}
\gtab{G}{320003}
\psk


\beginverse
\[C] Как часто я смотрю на звёзды,
Наверное не просто
\[Am] Бросить всё и пустым 
Отправиться в космос.
\bqk

\[Em] Мне говорили люди, 
Просили стать серьёзным, 
\[G] Мол, я стал взрослым, 
И для мечты моей уже поздно.
\bqk

А я по--прежнему смотрю 
В это чёрное небо,
Там, где планеты
Бороздят просторы вселенной.
\bqk

И им наверное не надо 
Вставать в восемь утра.
Я не хочу на Землю, 
Я хочу туда.
\endverse


\beginchorus
Где ты мое Со$\oStar$.
Туда, где нет причин бояться завтра.
Туда, где сон, мой сон, вдруг станет правдой.
Туда, где все сердца прольются песнями.
\endchorus


\beginverse
А завтра всё, как всегда, 
Мы начнём сначала.
Я смотрю в телескоп, 
Укрывшись одеялом.
\bqk

И я верю, что, когда--нибудь, 
Настанет время,
Что я буду с вами. 
Я мечте своей верен.
\bqk

Тебя легко найти на небе, 
Моё Со$\oStar$.
Ещё мгновение, 
И будем вместе мы.
\bqk

Ещё минута, 
И сбудется моя мечта.
Я не хочу на Землю, 
Я хочу туда.
\endverse


\endsong

\beginscripture{}
Песня написана в 2012 году на 3-ей смене в лагере <<Бригантина+>> Машей Бякиной и Во\-вой Чеботарёвым. Она исполнялась на\linebreak ВП~СПбГПу в 2012 году в номинации <<Отряд\-ная>> самой Машей. 
\endscripture



\beginsong{Юлина Песня}[
  by={Юля Анисимова},
  sr={СПО <<Созвездие>>},
  cr={}
  ]


\ifdefined\EXPLAINED
	%{\nolyrics Бой: $\downarrow \quad \downarrow \uparrow \quad \downarrow \uparrow \downarrow$} %6-ка
	%\bigskip
	
\else 

\fi



\beginverse
Детский смех, или плач - 
А вожатскую не открыть:
Полупуст сладкий шкаф,
Куча нужных бумаг на полу.
Говорили же мне, что непросто 
Вожатым быть.
Говорили же мне, 
Что без лагеря жизнь ни к чему.
\endverse


\beginchorus
И вот на смене отряд,
И зелень курток вокруг.
А дети рядом стоят,
И ста сердец слышен стук.
И, словно маме родной,
Я доверяю тебе.
Ты будешь вечно со мной,
Моё Созвездие-е.
\endchorus


\beginverse
Третий час. Я не сплю.
Я пишу отчёт о трудном дне:
Кто сбежал, заболел,
И о том, как здесь хочется спать.
Но полсмены ещё
Предстоит отработать мне.
Где-то хлопнула дверь - 
Значит, нужно идти проверять.
\endverse


\beginchorus
Наверно, в Питере дождь
Стучит по крышам родных.
А у меня сто детей
И нет вообще выходных.
Но звёзды рядом мои -
И, значит, легче вдвойне,
Ведь греет светом любви
Моё Созвездие-е.
\endchorus

\beginverse
Вот и всё. Нам пора
На работу и в институт.
Не грусти, дорогой,
Будет лагерь ещё впереди.
Кто бы знал, кто бы знал,
Как вожатые смены ждут.
Ты, когда подрастёшь,
Тоже к нам в СПО приходи.
\endverse


\beginchorus
И станет шире наш круг
На пару искренних глаз.
И ты увидишь, что друг
Надёждный в каждом из нас.
И пару тайн передать
Смогу в наследство тебе,
И будет долго сиять
Моё Созвездие-е.
\bigskip

И я могу повторить
Второй и тысячный раз,
Что буду вечно любить
Улыбки всех ваших глаз,
Что, словно маме родной,
Я доверяю тебе.
Так будь же вечно со мной,
Моё Созвездие-е.
\endchorus

\endsong


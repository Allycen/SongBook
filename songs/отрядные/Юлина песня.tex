\beginsong{Юлина Песня}[
  by={Юля Анисимова},
  sr={СПО <<Со$\oStar$>>},
  cr={2011 г.}
  ]


{\nolyrics Бой: $\down\q   \downs \up\q   \up \downs \up $}
\ptk

{\nolyrics Перебор: Б---3---2---3---1---3---2---3}
\pchk


\gtab{Am}{X02210}
\gtab{F+7}{X03210}
\gtab{Dm}{XX0231}
\gtab{E}{022100}
\gtab{G}{320003}

\gtab{C}{X32010}
\gtab{E7}{020100}
\psk


\beginverse
Детский с\[Am]мех, или плач ---
А во\[F+7]жатскую не отк\[Am]рыть:
Полу\[Dm]пуст сладкий шкаф,
Куча \[E]нужных бумаг на полу.
\bqk

Говорили же мне, что непросто 
Вожатым быть.
Говорили же мне, 
Что без лагеря жизнь ни к чему.
\endverse


\beginchorus
И вот на смене от\[Am]ряд,
И зелень курток вокруг.
А дети рядом сто\[Dm]ят,
И ста сердец слышен стук.
И, словно маме род\[G]ной,
Я доверяю тебе.
Ты будешь вечно со м\[C]ной,
Моё Созвездие-\[E7]е.
\endchorus


\beginverse
Третий час. Я не сплю.
Я пишу отчёт о трудном дне:
Кто сбежал, заболел,
И о том, как здесь хочется спать.
Но полсмены ещё
Предстоит отработать мне.
Где--то хлопнула дверь ---
Значит, нужно идти проверять.
\endverse


\beginchorus
Наверно, в Питере дождь
Стучит по крышам родных.
А у меня сто детей
И нет вообще выходных.
Но звёзды рядом мои ---
И, значит, легче вдвойне,
Ведь греет светом любви
Моё Созвездие-е.
\endchorus

\beginverse
Вот и всё. Нам пора
На работу и в институт.
Не грусти, дорогой,
Будет лагерь ещё впереди.
Кто бы знал, кто бы знал,
Как вожатые смены ждут.
Ты, когда подрастёшь,
Тоже к нам в СПО приходи.
\endverse


\beginchorus
И станет шире наш круг
На пару искренних глаз.
И ты увидишь, что друг
Надёждный в каждом из нас.
И пару тайн передать
Смогу в наследство тебе,
И будет долго сиять
Моё Созвездие-е.
\bqk

И я могу повторить
Второй и тысячный раз,
Что буду вечно любить
Улыбки всех ваших глаз,
Что, словно маме родной,
Я доверяю тебе.
Так будь же вечно со мной,
Моё Созвездие-е.
\endchorus

\endsong


\beginscripture{}
Песня была написана на осенней смене 2011 года в ЗЦДЮТ <<Зеркальный>> Юлей Анисимовой. Слова были переложены на музыку Дашей Портновой. Куплеты играются перебором, припевы --- боем. 
\endscripture
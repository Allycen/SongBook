\beginsong{Лирическая}[
  sr = {СПО <<Со$\oStar$>>},
  txt = {Вика Завадская},
  musc = {В.Завадская, Л.Волков, Т.Ложкин, Д.Барабаш, К.Есипова}]

{\nolyrics Бой: $\down \down \down\ \down \down \down\ \down \down\q (\down\q   \downs \up\q   \ups \down \up) $} %6-ка
\pchk

\gtab{C}{X32010}
\gtab{G}{320003}
\gtab{Am}{X02210}
\gtab{D}{XX0232}
\gtab{A}{X02220}

\gtab{Hm}{2:X13321}
\psk


\beginverse*
{\nolyrics Вступление: }
\endverse
\plik


\begin{lilypond}
#(set-global-staff-size 16)
#(set! paper-alist (cons '("my size" . (cons (* 3.2 in) (* 2 in))) paper-alist))

\paper {
  #(set-paper-size "my size")
  indent = 0 
}

PartpZeroVoiceZero =  \relative e {
  \clef "None" \stopStaff \override Staff.StaffSymbol #'line-count =
  #6 \startStaff \key c \major \numericTimeSignature\time 4/4 \repeat
  volta 2 {
    <e c>4 \4 \f \5 \f c'4 \2 \f r2 \f | % 2
    \clef "None" \stopStaff \override Staff.StaffSymbol #'line-count
    = #6 \startStaff \key c \major \numericTimeSignature\time 4/4
    <b, g>4 \5 \f \6 \f d'4 \2 \f r2 \f | % 3
    \clef "None" \stopStaff \override Staff.StaffSymbol #'line-count
    = #6 \startStaff \key c \major \numericTimeSignature\time 4/4
    <e, a,>4 \4 \f \5 \f c'4 \2 \f r2 \f | % 4
    \clef "None" \stopStaff \override Staff.StaffSymbol #'line-count
    = #6 \startStaff \key c \major \numericTimeSignature\time 4/4
    <e, a,>4 \4 \f \5 \f b'4 \2 \f r2 \f
  }
  | % 5
  \clef "None" \stopStaff \override Staff.StaffSymbol #'line-count =
  #6 \startStaff \key c \major \numericTimeSignature\time 4/4 \repeat
  volta 2 {
    <e, c>4 \4 \f \5 \f g'4 \1 \f <b,, g>4 \5 \f \6 \f e'4 \2 \f | % 6
    \clef "None" \stopStaff \override Staff.StaffSymbol #'line-count
    = #6 \startStaff \key c \major \numericTimeSignature\time 4/4 <d
    a,>4 \2 \f \5 \f <b a,>4 \2 \f \5 \f <d a,>4 \2 \f \5 \f <b
    a,>4 \2 \f \5 \f
  }
}


   \new TabStaff {
     
   %\tabFullNotation
    \PartpZeroVoiceZero
   }

\end{lilypond}

\beginverse
\[C] Внезапно отк\[G]рыли эту д\[Am]верь ---
И вот мы с Со$\oStar$ теперь!
Сколько дождливых, тусклых, серых дней
Своей улыбкой согревали вы!
\endverse


\beginchorus
Е-\[C]е-е-е-\[G]е-е-е-\[Am]е-е \rep 2
\endchorus

\beginverse*
{\nolyrics Проигрыш: \[C G Am] \rep 2}
\endverse

\beginverse
Ваши улыбки, хохот и тепло
В моей душе живут давно
Хоть груз учёбы на плечах лежит,
Я мыслью с вами, Звёзды, всё равно!
\bqk

И сколько мы дней не встречались
Исправно ведём календарь,
Мы б в ваших объятьях остались,
Наверное, с февраля по май.
\endverse


\beginverse
\[D] Мы каждо\[A]му хотим ска\[Hm]зать:
<<Ты клёвый, боец, мы любим тебя!>>
Желаем страстно и мы такими стать,
Детские сердца своим жаром зажигать!
\bqk

Мне Звёзды мои рассказали,
Как справиться с сотней детей.
О лагере мы замечтали,
Нет мысли теперь нам милей!
\endverse

\endsong

\beginscripture{}
Песня написана на юбилей отряда в мае 2015 года Викой Завадской. Она была исполнина 19 мая кандидатами на творческом вечере в честь дня рождения Со$\oStar$.
\endscripture
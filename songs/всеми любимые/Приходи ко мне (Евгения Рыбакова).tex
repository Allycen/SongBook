\songcolumns{1}

\beginsong{Приходи ко мне}[
  by={Евгения Рыбакова}]


{\nolyrics \boi $\down\q   \down \up\q   \up \down \up $} %6-ка
\pchk


\gtab{D}{XX0232}
\gtab{Hm}{2:X13321}
\gtab{G}{320003}
\gtab{A}{X02220}
\psk


{\nolyrics Вступление: 
  \gtab{}{XX0233}
  \gtab{}{XX0232} \rep 2
  \gtab{}{X24432}
  \gtab{}{X24422}
  \gtab{}{X24432}

  \hspace{60pt}\gtab{}{3:133211}
  \gtab{}{3:131213} 
  \gtab{}{3:133211}
  \hspace{25pt}\gtab{}{X02220}
  \gtab{}{X02230}
  \gtab{}{X02220}
  \gtab{}{X02200}
  \gtab{}{X02220}
}
\pchk


\beginverse
\[D] Приходи ко м\[Hm]не --- у меня есть \[G]дом.
В доме есть ка\[A]мин и вино.
Приходи ко мне --- у меня есть пять
Стеллажей, забитых кино.
\bqk

Приходи ко мне --- приноси цветы,
Приноси росу на губах,
Приноси рассвет и охапки звёзд,
Остывающих в руках.
\endverse


\beginchorus
Приходи ко мне --- 
Я буду ждать тебя.

Приходи ко мне,
Ко мне \rep 4
\endchorus


\beginverse
Приходи ко мне --- будем жечь камин
И считать прохожих в кино.
Приходи ко мне --- будут на губах
Высыхать роса и вино.
\bqk

Приходи ко мне --- будем танцевать
И встречать рассвет на окне.
Из цветов и звёзд сложим новый день.
Просто приходи
Ко мне \rep 6
\bqk

Приходи ко мне\dots
\endverse

\endsong

\beginscripture{}
Песня играется с каподастром, зажатым на втором ладу. 
\endscripture

\songcolumns{2}
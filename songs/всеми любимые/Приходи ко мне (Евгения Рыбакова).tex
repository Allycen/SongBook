\songcolumns{1}

\beginsong{Приходи ко мне}[
  by={Евгения Рыбакова}]

{\nolyrics Бой: $\down\q   \down \up\q   \up \down \up $} %6-ка
\medskip

\gtab{Dsus4}{XX0233}
\gtab{D}{XX0232}
\gtab{B}{X24432}
\gtab{Bm}{X24422}
\gtab{A}{X02220}
\gtab{Asus4}{X02230}
\gtab{Asus2}{X02200}
\gtab{G}{3:133211}
\gtab{G9}{3:131213}
\medskip

\beginverse*
{\nolyrics Вступление: \[Dsus4 D Dsus4 D] \quad \[B Bm B] 
 \hspace{56pt} \[G G9 G G9] \quad \[A Asus4 A Asus2 A] }
\endverse

\beginverse
\[D] Приходи ко м\[Bm]не --- у меня есть \[G]дом.
В доме есть ка\[A]мин и вино.
Приходи ко мне --- у меня есть пять
Стеллажей забитых кино.
\medskip

Приходи ко мне --- приноси цветы,
Приноси росу на губах.
Приноси рассвет и охапки звёзд
Остывающих в руках.
\endverse


\beginchorus
Приходи ко мне
Я буду ждать тебя.

Приходи ко мне,
Ко мне \rep 4
\endchorus


\beginverse
Приходи ко мне --- будем жечь камин
И считать прохожих в кино.
Приходи ко мне --- будут на губах
Высыхать роса и вино.
\medskip

Приходи ко мне --- будем танцевать
И встречать рассвет на окне.
Из цветов и звёзд сложим новый день.
Просто приходи
Ко мне \rep 6
\medskip

Приходи ко мне\dots
\endverse

\endsong

\beginscripture{}
Песня играется с каподастром, зажатым на втором ладу. 
\endscripture

\songcolumns{2}
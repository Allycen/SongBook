\beginsong{Город--сказка}[
  by={Тынцы Минус}]

{\nolyrics Бой: $\down\q   \down \up\q   \up \down \up $}
\pchk

\gtab{Dm}{XX0231}
\gtab{B}{X13331}
\gtab{A}{X02220}
\gtab{Gm}{3:133111}
\psk

\beginverse*
{\nolyrics Вступление: \[Dm B Gm A] \rep 2}
\endverse

\beginverse
\[Dm]Я шагаю \[B]по проспекту, 
\[Gm]По ночному \[A]городу.
Я иду, потому что у меня есть ноги,
Я умею ходить, и поэтому иду.
\bigskip

Иду навстречу цветным витринам,
Мимо пролетают дорогие лимузины.
В них женщины проносятся с горящими глазами,
Холодными сердцами, золотыми волосами.
\endverse


\beginchorus
\[Gm]Город--сказка, \[A]город--мечта
Попа\[Dm]дая в его сети, пропа\[B]даешь навсегда.
Глотая его воздух простуд и сквозняков,
С запахом бензина и дорогих духов.
\endchorus


\beginverse
Звёзд на небе мало, но это не беда ---
Здесь почти что в каждом доме есть своя и не одна.
Электричество, газ, телефон, водопровод,
Коммунальный рай без хлопот и забот.
\endverse


\beginverse
Дым высоких труб, бег седых облаков
Нам укажет приближение холодных ветров.
Тянет солнечных лучей в паутине проводов
Над жестяными крышами обшарпанных домов.
\bigskip

Иду навстречу цветным витринам,
Мимо пролетают дорогие лимузины.
В них женщины проносятся с горящими глазами
Холодными сердцами, золотыми волосами.
\endverse


\endsong


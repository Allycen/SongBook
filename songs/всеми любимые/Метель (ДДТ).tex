\songcolumns{1}

\beginsong{Метель}[
  by={ДДТ}]

{\nolyrics \boi $\down\q   \down \up \down \up$} 
\ptk

{\nolyrics \pere 4---2---1---2-3} 
\pchk

\gtab{Em}{022000}
\gtab{C}{X32010}
\gtab{G}{320003}
\gtab{D}{XX0232}
\psk

\beginverse*
{\nolyrics Вступление: }
\endverse
\plik

\begin{lilypond}
#(set-global-staff-size 16)


PartpZeroVoiceZero =  \relative e, {
  \clef "None" \stopStaff \override Staff.StaffSymbol #'line-count =
  #6 \startStaff \key c \major \time 2/4 e8 \6 \mf g'8 \3 \mf <e' b>8
  \1 \mf \2 \mf b,16 \5 \mf e16 \4 \mf | % 2
  \clef "None" \stopStaff \override Staff.StaffSymbol #'line-count =
  #6 \startStaff \key c \major \time 2/4 c8 \5 \mf e16 \4 \mf g16 \3
  \mf g8 \4 \mf fis8 \4 \mf | % 3
  \clef "None" \stopStaff \override Staff.StaffSymbol #'line-count =
  #6 \startStaff \key c \major \time 2/4 g,4 \6 \mf g16 \6 \mf a16 \6
  \mf b16 \5 \mf c16 \5 \mf | % 4
  \clef "None" \stopStaff \override Staff.StaffSymbol #'line-count =
  #6 \startStaff \key c \major \time 2/4 d8 \4 \mf a'16 \3 \mf d16 \2
  \mf e16 \1 \mf d16 \2 \mf a8 \3 \mf | % 5
  \clef "None" \stopStaff \override Staff.StaffSymbol #'line-count =
  #6 \startStaff \key c \major \time 2/4 e,8 \6 \mf g'8 \3 \mf <e' b>8
  \1 \mf \2 \mf b,16 \5 \mf e16 \4 \mf | % 6
  \clef "None" \stopStaff \override Staff.StaffSymbol #'line-count =
  #6 \startStaff \key c \major \time 2/4 c8 \5 \mf e16 \4 \mf g16 \3
  \mf g8 \4 \mf fis8 \4 \mf | % 7
  \clef "None" \stopStaff \override Staff.StaffSymbol #'line-count =
  #6 \startStaff \key c \major \time 2/4 b4 \4 \mf b4 \4 \mf | % 8
  \clef "None" \stopStaff \override Staff.StaffSymbol #'line-count =
  #6 \startStaff \key c \major \time 2/4 d,16 \4 \mf a'16 \3 \mf d8 \2
  \mf e8 \1 \mf d16 \2 \mf a16 \3 \mf
}


   \new TabStaff {
     
    %\tabFullNotation
    \PartpZeroVoiceZero
   }

\end{lilypond}


\beginverse
\[Em] Коро\[C]нована лу\[G]ной, \[D] как начало --- высока,
Как победа --- не со мной, как надежда --- не легка.
\bqk

За окном стеной метель, жизнь по горло занесло,
Сорвало финал с петель, да поела все тепло.
\endverse


\beginchorus
Играй, как можешь, сыграй,
Закрой глаза и вернись,
Не пропади, но растай,
Да колее поклонись.
\bqk

Моё окно отогрей,
Пусти по полю весной,
Не доживи, но созрей,
И будешь вечно со мной, \rep 3
Со мной.
\endchorus

\beginverse*
{\nolyrics Проигрыш: \dittotikz \ , {\comm как во вступлении}}
\endverse


\beginverse
Ищут землю фонари, к небу тянется свеча,
На снегу следы зари --- крылья павшего луча.
\bqk

Что же, вьюга, наливай, выпьем время натощак,
Я спою --- ты в такт пролай о затерянных вещах.
\endverse


\beginverse
Осторожно, не спеша, с белым ветром на груди,
Где у вмёрзшей в лёд ладьи ждёт озябшая душа.
\endverse

\endsong

\beginscripture{}
Первые половины всех куплетов играются соло партией из вступления, вторые половины --- боем. Припев играется перебором. Песня исполнялась на ВП СПбГПу в 2012 году в номинации <<Исполни\-тель>> нашим бойцом Мишей Пономарёвым. 
\endscripture

\songcolumns{2}
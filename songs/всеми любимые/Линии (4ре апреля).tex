\songcolumns{1}

\beginsong{Линии}[
  by={4ре апреля}]


{\nolyrics \boi $\down \down \ \down\q  \down \up \down \down \ \down$} 
\pchk

\gtab{Asus2}{X02200}
\gtab{H5}{2:X13300}
\gtab{E}{022100}
\gtab{C#5}{4:X13300}
\gtab{A}{5:133200}
\gtab{H}{7:133200}
\gtab{E5}{7:X133XX}
\gtab{E5*}{7:X132XX}
\gtab{F#m}{2:133111}
\psk


\beginverse*
{\nolyrics Вступление: }
\endverse
\plik

\begin{lilypond}
#(set-global-staff-size 16)
#(set! paper-alist (cons '("my size" . (cons (* 2.5 in) (* 2 in))) paper-alist))


PartpZeroVoiceZero =  \relative a, {
  \stopStaff \override Staff.StaffSymbol #'line-count =
  #6 \startStaff \key c \major \numericTimeSignature\time 4/4 a8 \5
   a8 \5  e''8 \3  a,,8 \5  a8 \5  dis'8 \3  a,8 \5
   dis'8 \3  | % 2
  \stopStaff \override Staff.StaffSymbol #'line-count =
  #6 \startStaff \key c \major \numericTimeSignature b,8 \6
   b8 \6  e'8 \3  b,8 \6  b8 \6  dis'8 \3  b,8 \6
   dis'8 \3  | % 3
  \stopStaff \override Staff.StaffSymbol #'line-count =
  #6 \startStaff \key c \major \numericTimeSignature e,8 \5
   e8 \5  e'8 \3  e,8 \5  e8 \5  dis'8 \3  e,8 \5
   e8 \5  | % 4
  \stopStaff \override Staff.StaffSymbol #'line-count =
  #6 \startStaff \key c \major \numericTimeSignature fis'4 \3
   e,8 \5  e'8 \3  e,8 \5  e8 \5  dis'8 \3  b8 \4
   | % 5
  \stopStaff \override Staff.StaffSymbol #'line-count =
  #6 \startStaff \key c \major \numericTimeSignature a,8 \5
   a8 \5  e''8 \3  a,,8 \5  a8 \5  dis'8 \3  a,8 \5
   dis'8 \3  | % 6
  \stopStaff \override Staff.StaffSymbol #'line-count =
  #6 \startStaff \key c \major \numericTimeSignature b,8 \6
   b8 \6  e'8 \3  b,8 \6  b8 \6  dis'8 \3  b,8 \6
   dis'8 \3  | % 7
  \stopStaff \override Staff.StaffSymbol #'line-count =
  #6 \startStaff \key c \major \numericTimeSignature cis,8 \6
   cis8 \6  e'8 \3  cis,8 \6  cis8 \6  dis'8 \3 
  cis,8 \6  cis8 \6  | % 8
  \stopStaff \override Staff.StaffSymbol #'line-count =
  #6 \startStaff \key c \major \numericTimeSignature fis'4 \2
   cis,8 \6  e'8 \3  cis,8 \6  cis8 \6  dis'8 \3  b8
  \4 
}

   \new TabStaff {
     
    %\tabFullNotation
    \PartpZeroVoiceZero
   }

\end{lilypond}


\beginverse
На твоём окне \[Asus2]инеем зи\[H5]ма рисует \[E]линии.
Знай, тепло твоего \[Asus2]имени нико\[H5]гда не растает вну\[C#5]три меня.
Когда на город о\[A]пустится снег, мы \[H]обязательно увидим\[E]ся во сне.
Всего этого \[A]мира нет, \[H]если ты во \[E5 (E5*)]мне.
\endverse


\beginchorus
Но ты смеёшься\dots \[F#m]Осталось \[A]прикосно\[H]венье, и в\[E]сё взорвётся.
\[C#5] Я знаю, \[A] между нами \[H] ничего не бы\[E]вает просто.
Если придётся, \[A] \[H]мы разобьёмся в\[E]месте и 
Го\[C#5]рим ярче солнца, \[A] ос\[H]танемся в нашей \[E]вечности.
\endchorus


\beginverse
Зимний холод закружится, cделав всех злыми и равнодушными.
Проникая глубже в душу мне, но не думай, что всё вокруг рушится.
Даже из самых тёмных дней мы возьмём только самое лучшее.
В ожидании летних дней сердце точно забьётся быстрей.
\endverse


\beginchorus
Люди --- ну\[Asus2]ли, и мы среди \[H5]них в толпе, но как \[E]будто одни. \[C#5]
На твоём окне \[A]инеем зима чертит \[H]линии --- то самое \[C#5]имя, смотри.
Если придётся, мы разобьёмся вместе и
Горим ярче солнца, останемся в нашей вечности.
\endchorus

\endsong

\beginscripture{}
Спасибо Денису Пронину из ССО <<Монолит>> за подбор аккордов и соло партий, за то, что сделал песню популярной среди политеховских студенческих отрядов, благодаря своему незабываемому исполнению.
\endscripture

\songcolumns{2}
\beginsong{Мы никогда не умрём}[
  by={К. Комаров, М. Башаков, К. Арбенин}]


{\nolyrics Бой: $\down\q   \down \up\q   \up \down \up $}
\pchk

\gtab{Em}{022000}
\gtab{C}{X32010}
\gtab{D}{XX0232}
\gtab{B}{X24442}
\gtab{Am}{X02210}
\psk

\beginverse
\[Em]Мы сегодня встали рано, вс\[C]тали затемно.
\[D]Люди били в барабаны, \[Em]выли матерно.
То ли общая тревога, то ли празднество.
Нам с тобою, слава богу, то без разницы.

\[B] Когда мы вместе, \[C] когда мы поём, 
\[D] Такое чувство, \[Em] что мы никогда не умрём!
\endverse


\beginverse
Дует ветер ледяной в нашу сторону,
И кричат над головой птицы--вороны.
Отчего же, отвечай, нам так весело?
Просто песня ту печаль перевесила!
Когда мы вместе, когда мы поём,
Такое чувство, что мы никогда не умрём!
\endverse


\beginchorus
\[Am] Это \[Em]больше, чем \[Am]я, это \[Em]больше, чем \[Am]ты,
Это \[C]вечной сво\[G]боды дур\[B]манящий в\[Em]дох!
Это наша любовь, это наши мечты,
Это с неба тебе улыбается бог! 
\endchorus


\beginverse
У меня сейчас внутри бочка пороха,
Только спичку поднеси --- будет шороху!
А башка моя сама в петлю просится ---
То, что сводит нас с ума, то и по сердцу!
Когда мы вместе, когда мы поем,
Такое чувство, что мы никогда не умрём!
\endverse

\beginchorus
\dittotikz \ + \dots
Это больше, чем я, это больше, чем ты,
Это тёплое солнце и ночью, и днём!
Это наша любовь, это наши мечты,
И поэтому мы никогда не умрём! \rep 2
\endchorus

\beginverse*
{\nolyrics Проигрыш: \[Em C D Em] \rep 4 }
\endverse


\beginverse
Когда мы вместе, когда мы поём,
Такое чувство, такое чувство!
\endverse



\beginchorus
\dittotikz \ + \dots
Это больше, чем я, это больше, чем ты,
Это тёплое солнце и ночью, и днём!
Это наша любовь, это наши мечты,
И поэтому мы никогда не умрём! \rep 2
\endchorus


\beginverse*
{\nolyrics Проигрыш: \[Em C D Em] \rep 2 }
\endverse

\beginverse
Мы никогда не ум\dots, мы никогда не ум\dots
Мы никогда не умрём! \rep 8
\endverse



\endsong

\beginscripture{}
Оригинальная версия этой песни играется на тон ниже, но в отрядах её привыкли играть в тональности Ми--минор.
Эта песня исполнялась на ВП Лэти 2012 года в номинации <<дуэт>> Михаилом Душкиным и Катей Крайнеевой из СПО <<Алые паруса>>. 
\endscripture

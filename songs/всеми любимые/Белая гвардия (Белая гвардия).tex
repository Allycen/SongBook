\beginsong{Белая гвардия}[
  by={Белая гвардия}]


{\nolyrics \boi $\down\q   \down \up\q   \up \down \up $}
\pchk

\gtab{Am}{002210}
\gtab{D}{XX0232}
\gtab{G}{320003}
\gtab{C}{X32010}
\gtab{Am}{002210}

\gtab{H7}{X21202}
\gtab{Em}{022000}
\gtab{D7}{XX0212}
\gtab{Am7}{002010}
\psk


\beginverse*
{\nolyrics Вступление: }
\endverse
\plik

\begin{lilypond}
#(set-global-staff-size 16)
#(set! paper-alist (cons '("my size" . (cons (* 3.2 in) (* 2 in))) paper-alist))

% если песня в ДВЕ колонки
\paper {
  #(set-paper-size "my size")
  indent = 0
}

PartpZeroVoiceZero =  \relative d'' {
  \clef "None" \stopStaff \override Staff.StaffSymbol #'line-count =
  #6 \startStaff \key c \major \numericTimeSignature\time 4/4 d8 \1
  \mf c4 \1 \mf e,2 \1 \mf r8 \mf | % 2
  \clef "None" \stopStaff \override Staff.StaffSymbol #'line-count =
  #6 \startStaff \key c \major \numericTimeSignature\time 4/4 c4 \3
  \mf d8 \3 \mf e4 \2 \mf fis8 \2 \mf g4 \2 \mf | % 3
  \clef "None" \stopStaff \override Staff.StaffSymbol #'line-count =
  #6 \startStaff \key c \major \numericTimeSignature\time 4/4 c8 \1
  \mf b4 \1 \mf d,2 \3 \mf r8 \mf | % 4
  \clef "None" \stopStaff \override Staff.StaffSymbol #'line-count =
  #6 \startStaff \key c \major \numericTimeSignature\time 4/4 c4 \3
  \mf d8 \3 \mf e4 \2 \mf fis8 \2 \mf g4 \2 \mf | % 5
  \clef "None" \stopStaff \override Staff.StaffSymbol #'line-count =
  #6 \startStaff \key c \major \numericTimeSignature\time 4/4 b8 \1 \f
  a4 \1 \f c,2 \3 \f r8 \f | % 6
  \clef "None" \stopStaff \override Staff.StaffSymbol #'line-count =
  #6 \startStaff \key c \major \numericTimeSignature\time 4/4 a4 \3
  \mf b8 \3 \mf cis4 \2 \mf dis8 \2 \mf e4 \2 \mf | % 7
  \clef "None" \stopStaff \override Staff.StaffSymbol #'line-count =
  #6 \startStaff \key c \major \numericTimeSignature\time 4/4 g2 \1
  \mf r2 \mf
}


   \new TabStaff {
    %\tabFullNotation
    \PartpZeroVoiceZero
   }

\end{lilypond}


\beginverse
\[Em]Белая гвардия, белый снег
\[Am7]Белая музыка революции,
\[D7]Белая женщина, нервный смех,
\[G]Белого платья слегка коснуться.
\bqk

Белой рукой распахнуть окно,
Белого света в нём не видя,
Белое выпить до дна вино,
В красную улицу в белом выйти.
\endverse


\beginchorus
Ког\[Em]да ты вернёшься ---
Всё будет и\[Am]наче и нам не узнать друг друга,
Ког\[D7]да ты вернёшься, 
А я не же\[G]на и даже не подруга.
\bqk

Ког\[C]да ты вернёшься ко мне, 
Так бе\[Am]зумно тебя любившей в прошлом,              
Ког\[H7]да ты вернёшься ---
Увидишь, что ж\[Em]ребий давно и не нами брошен.
\endchorus

\beginverse*
{\nolyrics Проигрыш: \[Am] \[D] \[G] \[C] \[Am] \[H7] \[Em]}
\endverse

\beginverse
Сизые сумерки прошлых лет
Робко крадутся по переулкам.
В этом окне еле брезжит свет.
Ноты истерзанны, звуки гулки.
\bqk

Тонкие пальцы срывают аккорд.
Нам не простят безрассудного дара.
Бьются в решетку стальных ворот
Пять океанов земного шара.
\endverse


\beginverse
Красный трамвай простучал в ночи.
Красный закат догорел в бокале.
Красные--красные кумачи
С красных деревьев на землю пали.
\bqk

Я не ждала тебя в Октябре,
Виделись сны, я листала сонник.
Красные лошади на заре
Бились копытами о подоконник.
\endverse


\beginchorus
Когда ты вернёшься --- 
Всё будет иначе, и нам не узнать друг друга,
Когда ты вернёшься, 
А я не жена и даже не подруга.
\bqk

Когда ты вернёшься, 
Вернёшься в наш город обетованный.
Когда ты вернёшься, 
Такой невозможный и такой желанный.
\endchorus

\endsong







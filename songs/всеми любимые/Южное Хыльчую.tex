\beginsong{Южное Хыльчую}[
  by={Антон Захаров},
  sr={ССО <<Монолит>>},
  cr={}
  ]


{\nolyrics Бой №1: $\down \up \downx \up\q   \up \downx \up $}
\ptk

{\nolyrics Бой №2: $\down\q   \down\q   \up\ \up \down\q   \down \up \down \up$} 
\pchk

\gtab{Cadd9}{X32033}
\gtab{Em7}{022033}
\gtab{G}{320003}
\gtab{D}{XX0232}
\psk

\beginverse*
{\nolyrics Вступление: \[Cadd9 Em7 G]
\hspace{62pt} \[Cadd9 Em7 D] }
\endverse

\beginverse
\[Cadd9] Бери. \[Em7] \[G]Больше.
\[Cadd9] Кидай. \[Em7] \[D]Дальше.
Это --- Монолит.
Ничего не делает --- просто стоит.
\endverse


\beginchorus
\lrep\ \[Cadd9]Южное \[G]Хыльчу\[Cadd9]и. 
\[Cadd9]Южное \[D]Хыльчу\[G]ю. \rrep\ \rep 2
\endchorus


\beginverse*
{\nolyrics Проигрыш: \dittotikz \ , {\comm как во вступлении} }
\endverse


\beginverse
Каска, тулуп и очки:
Как наши будни нелегки.
Копай, копай --- не стой!
Копай, но помни о той, 
Что дома ждет тебя!
\endverse


\beginverse
Копай, копай, копай\dots \rep {32} 
\endverse

\endsong

\beginscripture{}
Песня написана в 2010 году в посёлке <<Юж\-ное Хыльчую>> Антоном Захаровым.
Она ис\-по\-нялась на ВП СПбГПу в 2012 году в номи\-на\-ции <<Отрядная>> Антоном Захаровым и бой\-ца\-ми всех строительных отрядов, при\-сутствовавших в зале.
Третий куплет ис\-пол\-няется боем №1, вся остальная песня --- боем №2. 
\endscripture
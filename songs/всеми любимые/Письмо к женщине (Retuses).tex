\beginsong{Письмо к женщине}[
  by={Retuses},
  cr={<<Письмо к женщине>>, С.А. Есенин, 1924 г.}]

{\nolyrics Бой: $\down\ \down\ \down\ \down\ \down\ \down\ \down\ \down \up$} 
\medskip

\gtab{G}{3:133211}
\gtab{F#}{2:133211}
\gtab{Bm}{X24432}
\gtab{D}{XX0232}
\medskip


\beginverse*
{\itshape
Вы \[G]помните,
Вы в\[F#]сё, конечно, помните,
Как \[Bm]я стоял,
Приб\[D]лизившись к стене,
Взволнованно ходили вы по комнате
И что--то резкое
В лицо бросали мне.
\bigskip	

Вы говорили:
Нам пора расстаться,
Что вас измучила
Моя шальная жизнь,
Что вам пора за дело приниматься,
А мой удел ---
Катиться дальше, вниз.
\bigskip	

Любимая!
Меня вы не любили.
Не знали вы, что в сонмище людском
Я был, как лошадь, загнанная в мыле,
Пришпоренная смелым ездоком.
\bigskip	

Не знали вы,
Что я в сплошном дыму,
В разворочённом бурей быте
С того и мучаюсь, что не пойму ---
Куда несёт нас рок событий.
\bigskip	

Лицом к лицу
Лица не увидать.
Большое видится на расстоянье.
Когда кипит морская гладь,
Корабль в плачевном состоянье.
\bigskip

Земля---корабль!
Но кто--то вдруг
За новой жизнью, новой славой
В прямую гущу бурь и вьюг
Её направил величаво.
\bigskip

Ну кто ж из нас на палубе большой
Не падал, не блевал и не ругался?
Их мало, с опытной душой,
Кто крепким в качке оставался.
\bigskip
	
Тогда и я
Под дикий шум,
Но зрело знающий работу,
Спустился в корабельный трюм,
Чтоб не смотреть людскую рвоту.
Тот трюм был ---
Русским кабаком.
И я склонился над стаканом,
Чтоб, не страдая ни о ком,
Себя сгубить
В угаре пьяном.
\bigskip

\dots
\bigskip

Простите мне\dots
Я знаю:  вы не та ---
Живёте вы
С серьёзным, умным мужем;
Что не нужна вам наша маета,
И сам я вам
Ни капельки не нужен.
\bigskip	

Живите так,
Как вас ведёт звезда,
Под кущей обновлённой сени.
С приветствием,
Вас помнящий всегда
Знакомый ваш
\quad Сергей Есенин.}
\endverse

\endsong


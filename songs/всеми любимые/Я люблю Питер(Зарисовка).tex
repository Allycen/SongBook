\beginsong{Я люблю Питер}[
  by={Зарисовка}]


{\nolyrics Бой №1: $\down\q   \down\q   \down \up\ \up \down\q   \down \up \down \up$} 
\ptk

{\nolyrics Бой №2: $\down \up \down \up\q   \up \down \up $}
\pchk

\gtab{G}{320003}
\gtab{C}{X32010}
\gtab{D}{XX0232}
\psk


\beginverse
\[G]Моя песня не знает границ,
\[C]Помогает на с\[D]вет появиться.
Не выносит напуганных лиц,
И грозится во сне мне присниться.
\bqk

Жизнь проходит, меняя хиты,
Только Марли стоит вечно в моде.
Мои цели не в рамках мечты,
Они гордые птицы на воле.
\endverse


\beginchorus
\[G]Небо плывёт, \[C]солнце в зените.
Нена\[G]вижу Москву, \[D]я люблю Питер.
Захлебнуться хочу в море удачи.
Моё сердце поёт --- я не могу жить иначе. \rep 2
\endchorus


\beginverse
Я за ноту отдам целый свет,
Я за мысли куплю пол куплета.
Разорву мегаполисов сеть,
Чтобы жить на обочине лета.
\bqk

Разукрашу мелодии, сны,
И наклею улыбки на лица.
Я с Ямайкой общаюсь на ты,
Хоть она мне уж в мамы годится.
\endverse


\beginverse
Море прячет ошибки на дне,
Музыка --- под фанерой дешёвой.
Мы поём на одном языке,
На мелодии, в общем, драйвовой.
\bqk

Каждый может пробиться наверх,
И фантазией раздвинуть границы.
Только нужно поверить в успех,
Чтоб талант ваш не смог затаиться.
\endverse


\endsong

\beginscripture{}
Куплеты играются боем №1, припевы --- 
\Bcolon боем №2.
\endscripture
\beginsong{В платье белом}[
  by={Ляпис Трубецкой}]

{\nolyrics Бой: $\down\q   \downs \up\q   \up \down \up $}
\pchk

\gtab{Dm}{XX0231}
\gtab{G}{320003}
\gtab{C}{X32010}
\gtab{Am}{X02210}
\gtab{F}{133211}

\gtab{Fm}{133111}
\psk


\beginverse
\[Dm] Когда зимой хо\[G]лодною 
В кре\[C]щенские мо\[Am]розы,
Щебечет песню соловей, 
И распускаются мимозы,
\bqk

Когда взлетаешь к небесам 
И там паришь, пугая звезды,
А над окном твоим совьют 
Какие--нибудь птицы гнезда.
\bqk

Когда девчонка толстая 
Журнал приобретает <<Мода>>,
И снит, как будто юноши 
Eй в школе не дают прохода,
\bqk

Когда милиционер усатый 
Вдруг улыбнется хулигану
И поведет его под руки, 
Но не в тюрьму, а к ресторану.
\endverse


\beginchorus
З\[F]най, это лю\[Fm]бовь, 
C ней рядом А\[Am]мур крыльями машет,
З\[F]най, это лю\[Fm]бовь, 
Cердце не п\[C]рячь, Амур не про\[G]мажет.
\endchorus


\beginverse
Когда мальчишка на асфальте 
Мелом пишет чьё--то имя,
Когда телёнок несмышленый 
Губами ищет мамки вымя,
\bqk

Когда весёлый бригадир 
Доярку щиплет возле клуба,
Когда солдатик лысенький 
Во сне целует друга губы,
\bqk

Когда безродная дворняга 
Взобраться хочет на бульдога,
Когда в купаловскую ночь 
Две пары ног торчат из стога,
\bqk

Когда седой профессор под 
Дождем по лужам резво скачет,
А зацелованная им 
Девчонка над пятёркой плачет.
\endverse


\beginverse
\lrep\ \[Dm] Любовь зи\[F]мой 
При\[G]ходит в платье \[Am]белом,
Весной любовь приходит 
В платье голубом,
\bqk

Любовь приходит летом 
В платьице зелёном,
А осенью любовь 
Приходит в золотом. \rrep\ \rep 2
\endverse


\endsong


\beginsong{Песенка о медведях}[
  by={к/ф <<Кавказская пленница>>}]


{\nolyrics Бой: $\down\q   \down \up\q   \up \down \up $} %6-ка
\pchk

\gtab{Am}{X02210}
\gtab{Dm}{XX0231}
\gtab{E}{022100}
\gtab{C}{X32010}
\gtab{G}{320003}
	
\gtab{E7}{020100}
\gtab{A7}{X02020}
\psk


\beginverse
Г\[Am]де--то на белом с\[Dm]вете,
\[E]Там, где всегда мо\[Am]роз,
Трутся спиной мед\[Dm]веди
\[G]О земную \[C]ось.
\bqk

\[Dm]Мимо плывут сто\[G]летия,
C\[C]пят подо льдом мо\[Am]ря,
Т\[Am]рутся об ось мед\[Dm]веди,
\[E7]Вертится зем\[Am]ля.
\endverse


\beginchorus
\[Am A7]Ля\dotsля--ля--ля--\[Dm]ля\dotsля--ля, 
\[E]Вертится быст\[Am]рей земля \rep{2}
\endchorus


\beginverse
Крутят они, стараясь,
Вертят земную ось,
Чтобы влюблённым раньше
Встретиться пришлось.
\bqk

Чтобы однажды утром,
Раньше на год иль два,
Кто--то сказал кому--то
Главные слова.
\endverse


\beginverse
Вслед за весенним ливнем
Раньше придёт рассвет,
И для двоих счастливых
Много--много лет
\bqk

Будут сверкать зарницы,
Будут ручьи звенеть,
Будет туман клубиться,
Белый, как медведь.
\endverse


\endsong








\beginsong{Песня друзей}[
  sr = {м/ф <<Бременские музыканты>>},
  txt = {Ю.Энтин},
  musc = {Г.Гладкова}]



{\nolyrics \boi $\down\q   \down \up\q   \up \down \up $} %6-ка
\pchk

\gtab{C}{X32010}
\gtab{B}{X13331}
\gtab{Eb}{6:X13331}
\gtab{Em}{022000}
\gtab{G}{320003}
	
\gtab{Am}{X02210}
\gtab{Dm}{XX0231}
\gtab{F}{133211}
\psk


\beginverse*
{\nolyrics Вступление: \[C B C Eb] \rep 2} 
\endverse

\beginverse
\[C]Ничего на свете лучше \[Em]нету,
\[F]Чем бродить друзьям по белу с\[G]вету.
\[C]Тем, кто д\[Em]ружен, \[Am]не страшны тревоги!
\[F]Нам любые \[G]дороги до\[C]роги. \[(Am)] \rep 2
\endverse


\beginchorus
Ла--ла--ла--\[Am]ла, ла--ла\dots
Ла--ла--ла--\[Dm]ла --ла\dots ла--ла--ла, 
\[B]Е --- е--е, е--е!
\endchorus

\beginverse*
{\nolyrics Проигрыш: \[C B C Eb] \rep 2} 
\endverse


\beginverse
Мы своё призванье не забудем.
Смех и радость мы приносим людям!
Нам дворцов заманчивые своды
Не заменят никогда свободы. \rep 2
\endverse


\beginverse
Наш ковёр --- цветочная поляна.
Наши стены --- сосны--великаны.
Наша крыша --- небо голубое.
Наше счастье --- жить такой судьбою. \rep 2
\endverse


\endsong


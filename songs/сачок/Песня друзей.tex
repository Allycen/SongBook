\beginsong{Песня друзей}[
  by={м/ф <<Бременские музыканты>>}]


\ifdefined\EXPLAINED
	{\nolyrics Бой: $\down\q   \down \up\q   \up \down \up $} %6-ка
	\medskip

	\gtab{C}{X32010}
	\gtab{Bb}{X13331}
	\gtab{Eb}{6:X13331}
	\gtab{Em}{022000}

	\gtab{G}{320003}
	\gtab{Am}{X02210}
	\gtab{Dm}{XX0231}
	\gtab{F}{133211}
	\medskip
\else 

\fi

\beginverse*
{\nolyrics Вступление: \[C Bb C Eb] \rep 2} 
\endverse

\beginverse
\[C]Ничего на свете лучше \[Em]нету,
\[F]Чем бродить друзьям по белу \[G]свету.
\[C]Тем, кто \[Em]дружен, \[Am]не страшны тревоги!
\[F]Нам любые \[G]дороги до\[C]роги. \[(Am)] \rep 2
\endverse


\beginchorus
Ла--ла--ла--\[Am]ла, ла--ла,
Ла--ла--ла--\[Dm]ла --ла, ла--ла--ла, 
\[Bb]Е --- е--е, е--е!
\endchorus

\beginverse*
{\nolyrics Проигрыш: \[C Bb C Eb] \rep 2} 
\endverse


\beginverse
Мы своё призванье не забудем.
Смех и радость мы приносим людям!
Нам дворцов заманчивые своды
Не заменят никогда свободы. \rep 2
\endverse


\beginverse
Наш ковёр --- цветочная поляна.
Наши стены --- сосны--великаны.
Наша крыша --- небо голубое.
Наше счастье --- жить такой судьбою. \rep 2
\endverse


\endsong


\beginsong{Непогода}[
  by={}]


\ifdefined\EXPLAINED

	{\nolyrics Бой: $\down\q   \down \up\q   \up \down \up $} %6-ка
	\medskip

	\gtab{C}{X32010}
	\gtab{G}{320003}
	\gtab{E7}{020100}
	\gtab{Am}{X02210}
	\gtab{D}{XX0232}
	
	\gtab{Dm7}{XX0211}
	\gtab{G7}{320001}
	\gtab{F}{133211}
	\medskip
\else 

\fi


\beginverse
\[C]Изменения в природе
\[F]Происходят \[G]год от года.
\[C]Непогода нынче в моде,
\[F]Непогода, \[E7]непогода.
\medskip

\[Am]Словно из водопровода
Льёт на нас с не\[D]бес вода\dots
\endverse


\beginchorus
Пол\[F]года пло\[G]хая по\[C]года.
Полгода --- совсем никуда.
Полгода плохая погода.
Полгода --- сов\[E7]сем нику\[Am]да.
\medskip

Никуда, нику\[Dm7]да нель\[G7]зя ук\[C]рыться \[Am]нам,
Но откладывать жизнь никак нельзя.
Никуда, никуда, но знай, что где-то там
Кто-то ищет тебя сре\[E7]ди дож\[Am]дя.
\endchorus


\beginverse
Грома грозные раскаты
От заката до восхода.
За грехи людские плата ---
Непогода, непогода.
\medskip

Не ангина, не простуда ---
Посерьёзнее беда\dots
\endverse


\endsong

\beginscripture{}
Оригинальная версия этой песни играется на тон выше, но для простоты на сачках её играют в тональности До--мажор.
\endscripture





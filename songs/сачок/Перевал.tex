\beginsong{Перевал}[
  by={}]


{\nolyrics Бой: $\down\q   \down \up\q   \up \down \up $} %6-ка
\pchk

\gtab{Am}{X02210}
\gtab{Dm}{XX0231}
\gtab{E7}{020100}
\gtab{G7}{320001}
\gtab{C}{X32010}

\gtab{A7}{X02020}
\psk


\beginverse               
\[Am] Просто нечего нам \[Dm]больше терять,      
\[E7] Всё нам вспомнится на ст\[Am]рашном суде.
Эта ночь легла, как \[Dm]тот перевал,             
\[G7] За которым испол\[C]ненье надежд.
\bqk

\[A7] Видно прожитое п\[Dm]рожито зря --- не зря,
\[G7] Но не в этом, пони\[C]маешь ли, соль.
\[Am] Слышишь --- падают дож\[Dm]ди октября,                    
\[E7] Видишь --- старый дом сто\[Am]ит средь лесов.
\endverse


\beginverse
Мы затопим в доме печь, в доме печь,
Мы гитару позовем со стены.
Просто нечего нам больше беречь,
Ведь за нами все мосты сожжены.
\bqk

Все мосты, все перекрестки дорог,
Все прошёптанные тайны в ночи.
Каждый сделал всё, что мог, всё, что мог,
Но об этом помолчим, помолчим.
\endverse


\beginverse
И луна взойдёт оплывшей свечой,
Ставни скрипнут на ветру, на ветру.
Ах, как я тебя люблю горячо,
Годы это не сотрут, не сотрут.
\bqk

Мы оставшихся друзей созовём,
Мы набьём картошкой старый рюкзак,
Люди спросят, что за шум, что за гам,
Мы ответим --- просто так, просто так.
\endverse


\beginverse
$\dittoci$ {\comm первого куплета}
\endverse



\endsong


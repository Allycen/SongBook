\beginsong{Гимн ССО Лэти}[
   by = {В.Абарбанель}]



{\nolyrics \boi $\down \up \downx \up\q  \down \up \downx \up\q  (\down\q   \downx \up\q   \up \down \up)$} 
\pchk

\gtab{Em}{022000}
\gtab{A}{X02220}
\gtab{G}{320003}
\gtab{F}{133211}
\gtab{H}{2:X13331}

\gtab{C}{X32010}
\gtab{Am}{X02210}
\gtab{D}{XX0232}
\gtab{F#}{2:133211}
\psk


\beginverse*
Вступление: \[Em]На\dots--на--на--на--\[A]на--на--на\dots 
\hspace{57pt} на--\[G]на\dots --на--на--на--\[F]на--на--\[H]на.
\endverse

\beginverse
В\[Em]сё, окончен пос\[G]ледний эк\[A]замен.
Вновь реет отрядное знамя.
В\[C]новь дорога зо\[G]вёт нас туда,
Где \[F]сильные руки нуж\[H]ны.
\bqk

Пусть ищут другие славу ---
Нам всё по плечу и по нраву.
Дарит нам золотые лучи
Утро родной страны.
\endverse


\beginchorus
\[Em]Нам не з\[Am]нать покоя --- \[D]жизнь нем\[G]ного стоит,
\[C]Если \[F]ты не строил, \[F#]а сидел в теп\[H]ле, на--на--на.
Петь и дело делать, чтоб спина болела,
Чтобы вновь хотелось жить на этой зем\[Em]ле.
\endchorus


\beginverse
В путь, и все сомненья отбросим.
Мы легкой работы не просим.
Первый дом, как большая любовь,
Вечно помнится нам.
\bqk

Мы вдвое сильней если надо,
Нам счастье людское --- награда,
На мозолистых наших руках ---
Солнце будущих дней.
\endverse


\beginverse
Пусть нас караулят дипломы,
Но нам не сидеть больше дома:
Манит нас в неизвестную даль
Слово простое <<отряд>>.
\bqk

Пусть все это трудно и сложно,
Мы всюду придем и поможем,
Чтобы помнили люди всегда
Нас, Лэти, Ленинград.
\endverse


\endsong

\beginscripture{}
В середине третьего куплета после фразы 
\Bcolon <<Слово простое <<отряд>>>> 
\Acolonпринято кричать название своего отряда.
\endscripture



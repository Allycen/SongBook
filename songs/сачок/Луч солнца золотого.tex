\beginsong{Луч солнца золотого}[
  by={}]


{\nolyrics \boi $\down\q  \down\q  \up \ \up \down\q  \down \up \down \up$}
\pchk

\gtab{C}{X32010}
\gtab{Em}{022000}
\gtab{Am}{X02210}
\gtab{G7}{320001}
\gtab{G}{320003}

\gtab{C7}{032310}
\gtab{Dm}{XX0231}
\gtab{F}{133211}
\gtab{Fm}{133111}
\psk


\beginverse
\[C]Луч солнца золо\[Em]того
Ть\[Am]мы скрыла пеле\[C]на.
И между нами с\[Em]нова
Вд\[F]руг выросла сте\[Em]на.
\endverse

\beginverse*
Проигрыш: А--а--\[Am]а\dots   а--а--\[F]а\dots   а--а--\[G7]а
\endverse


\beginchorus
\[C] Ночь прой\[C7]дёт, наступит \[F]утро \[G]ясное,
\[C] Знаю, с\[Am]частье нас с то\[Dm]бой ж\[G]дёт.
\[C] Ночь прой\[C7]дёт, пройдёт по\[F]ра не\[Fm]настная,
\[C]Солнце взой\[Am]дёт\dots \[F G] \rep 2
\endchorus


\beginverse
Петь птицы перестали.
Свет звёзд коснулся крыш.
В час вьюги и печали
Ты голос мой услышь.
\endverse


\endsong

\beginscripture{}
Оригинальная версия этой песни играется на два тона выше, но для простоты на сач\-ках её играют в тональности 
\BcolonДо--мажор.
\endscripture
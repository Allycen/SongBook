\beginsong{17 лет}[
  by={Чайф}]


{\nolyrics Бой: $\down\q   \down \up\q   \up \down \up $} %6-ка
\pchk

\gtab{Am}{X02210}
\gtab{G}{320003}
\gtab{C}{X32010}
\gtab{C7}{032310}
\gtab{F}{133211}

\gtab{Fm}{133111}
\psk


\beginverse*
{\nolyrics Вступление: \[C C7 F Fm] \, \[C Am F G C]}
\endverse
\plik

\begin{lilypond}
#(set-global-staff-size 16)
#(set! paper-alist (cons '("my size" . (cons (* 2.5 in) (* 2 in))) paper-alist))

\paper {
  #(set-paper-size "my size")
}


PartpZeroVoiceZero =  \relative c {
  \clef "None" \stopStaff \override Staff.StaffSymbol #'line-count =
  #6 \startStaff \key c \major \numericTimeSignature\time 4/4 c4 \5
b4 \5  s2 | % 2
}

   \new TabStaff {
     
   %\tabFullNotation
    \PartpZeroVoiceZero
   }
\end{lilypond}



\beginverse
\[Am] Я \[F]вижу,
\[G]Я снова вижу те\[C]бя такой.
В дерзкой мини--юбке,
Что мой покой,
\bqk

Мой сон,  
Превратили, шутя,
В тебя ---
Я умоляю тебя.
\endverse


\beginchorus
\[C]Пусть всё будет \[C7]так, как ты захочешь.
\[F]Пусть твои глаза, как \[Fm]прежде, горят.
Я с тобой опять се\[Am]годня этой ночью,
Ну а впрочем, ну а в прочем,
Следующей ночью, следующей ночью,
Если захочешь,\[G] я опять у те\[C]бя.
\endchorus


\beginverse
Тебе семнадцать,
Тебе опять семнадцать лет.
Каждый твой день рожденья 
Хочет прибавить, а я скажу --- нет.
\bqk

Твой портрет, твои дети,
Я расскажу им о том:
<<Дети, вашей маме снова семнадцать,
Вы просто поверьте, а поймёте потом>>.
\endverse


\beginverse
Вазы в нашем доме, 
В них редко бывают цветы.
В мае снова будут тюльпаны, 
Я помню, их так любишь ты.
\bqk

Я напишу свою лучшую песню, 
Если будет угодно судьбе.
И первой ее сыграю тебе, 
Конечно, тебе.
\endverse

\endsong

\beginscripture{}
У этой песни есть известная переделанная версия из фильма <<Стиляги>>. Начинается она со слов:
\Bcolon <<Я вижу, я снова вижу глаза твои\dots>>.
\endscripture


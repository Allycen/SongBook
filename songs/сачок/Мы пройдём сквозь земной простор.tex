\beginsong{Мы пройдём сквозь земной простор}[
  by={}]


{\nolyrics Бой: $\down \up \downx \up\q  \down \up \downx \up\q  (\down\q   \downx \up\q   \up \downx \up)$} 
\pchk


\gtab{Em}{022000}
\gtab{Am}{X02210}
\gtab{H7}{X21202}
\gtab{C}{X32010}
\gtab{D}{XX0232}

\gtab{G}{320003}
\psk


\beginverse*
{\nolyrics Вступление: }
\endverse
\plik



\begin{lilypond}


#(set-global-staff-size 16)
#(set! paper-alist (cons '("my size" . (cons (* 3.2 in) (* 2 in))) paper-alist))

\paper {
  #(set-paper-size "my size")
  indent = 0 
}


PartpZeroVoiceZero =  \relative b {
  \clef "None" \stopStaff \override Staff.StaffSymbol #'line-count =
  #6 \startStaff \key c \major \time 8/4 b16 \3 b16 \3 b16 \3
  r16 e2 \2 r8 dis8 \2 e8 \2 fis8 \2 g4 \2
  b2 \1 | % 2
  \clef "None" \stopStaff \override Staff.StaffSymbol #'line-count =
  #6 \startStaff \key c \major  b,4 \3 c4. \3 c8 \3
  a4 \4 a4 \4 e'4 \2 e4 \2 dis4 \2 | % 3
}

   \new TabStaff {
     
   %\tabFullNotation
    \PartpZeroVoiceZero
   }
\end{lilypond}



\beginverse
Мы прой\[Em]дём сквозь земной простор 
По рав\[Am]нинам и пере\[H7]валам,
По вершинам огромных гор 
Сквозь земной простор небывалый.
\endverse


\beginchorus
Ведь не\[C]даром вста\[D]ёт за\[G]ря, 
Небосвод от зари весь красный. 
\lrep\ Только з\[Am]нать бы, что \[H7]всё не \[Em]зря, 
Только знать бы, что не напрасно. \rrep\ \rep 2
\endchorus


\beginverse
Не бывает на свете чудес, 
Всё мы сделаем своими руками. 
Города, что встают до небес, 
Никогда не построятся сами. 
\endverse


\beginverse
Мы пройдём сквозь земной простор, 
Будет много нас и будем мы вместе. 
Ну а те, что придут потом, 
Пусть подхватят вот эту песню.
\endverse


\endsong

\beginscripture{}
За основу данной песни было взят стих Эдуарда Асадова --- 
<<Я могу тебя очень ждать\dots>> 
в музыкальном исполнении.
\endscripture


\beginsong{Замыкая круг}[
 txt = {М.Пушкина},
 musc = {К.Кельми}]


{\nolyrics \boi $\down\q   \downs\q   \down \up \down \up$}
\pchk

\gtab{C}{X32010}
\gtab{F}{133211}
\gtab{Am}{X02210}
\gtab{Fm}{133111}
\gtab{G}{320003}

\gtab{Em}{022000}
\gtab{Dm}{XX0231}
\psk


\beginverse
\[C] Вот одна из \[F]тех историй, 
\[C] О которых \[Am]люди спорят
\[Fm] И не день не два, а много \[C]лет.

Началась она так просто --- 
Не с ответов, а с вопросов.
До сих пор на них ответов нет.
\bqk

Почему стремятся к свету
Все растения на свете?
Отчего к морям спешит река? 
Как мы в этот мир приходим? 
В чем секрет простых мелодий? 
Нам хотелось знать наверняка.
\endverse


\beginchorus
Замы\[F]кая к\[G]руг, 
Ты на\[Em]зад посмотришь вд\[Am]руг.
Там увидишь в \[Dm]окнах свет, 
Си\[G]яющий нам вс\[C]лед.
Пусть и\[F]дут дож\[G]ди, 
Прошлых \[Em]бед от них не ж\[Am]ди.
Камни пройден\[Dm]ных дорог 
Су\[F]мел пробить рос\[C]ток.
\endchorus


\beginverse
Открывались в утро двери, 
И тянулись вверх деревья,
Обещал прогноз то снег, то зной.
Но в садах рождённых песен 
Ветер лёгок был и весел,
И в дорогу звал нас за собой.
\endverse


\beginverse
Если солнце на ладони, 
Если сердце в звуках тонет,
Ты потерян для обычных дней.
Для тебя сияет полночь, 
И звезда спешит на помощь,
Возвращая в дом к тебе друзей.
\endverse

\beginverse
Свой мотив у каждой птицы, 
Свой мотив у каждой песни,
Свой мотив у неба и Земли.
Пусть стирает время лица, 
Нас простая мысль утешит --- 
Мы услышать музыку смогли.
\endverse

\endsong


\beginscripture{}
Оригинальная версия этой песни играется под фортепиано в тональности 
\BcolonСоль--мажор.
\endscripture
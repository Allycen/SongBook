\beginsong{Замыкая круг}[
  by={}]


\pchk

\gtab{C}{X32010}
\gtab{F}{133211}
\gtab{Fm}{133111}
\gtab{Em}{022000}
\gtab{Dm}{XX0231}

\gtab{G}{320003}
\psk


\beginverse
\[C]Вот одна из \[F]тех историй, 
\[C]О которых \[F]люди спорят
\[Fm]И не день не два, \[C]а много лет.


Началась она так просто 
Не с ответов, а с вопросов,
До сих пор на них ответов нет:
\bqk

Почему стремятся к свету
Все растения на свете?
Отчего к морям спешит река? 
Как мы в этот мир приходим? 
В чем секрет простых мелодий? 
Нам хотелось знать наверняка.
\endverse


\beginchorus
\[F]Замыкая \[G]круг, 
\[Em]Ты назад посмотришь \[Am]вдруг.
\[Dm]Там увидишь в окнах свет, 
\[G]Сияющий нам \[C]вслед.
\[F]Пусть идут \[G]дожди, 
\[Em]Прошлых бед от них не \[Am]жди.
\[Dm]Камни пройденных дорог 
\[G]Сумел пробить \[C]росток.
\endchorus


\beginverse
Открывались в утро двери, 
И тянулись вверх деревья,
Обещал прогноз то снег, то зной.
Но в садах, рожденных песней, 
Ветер легок был и весел,
И в дорогу звал нас за собой.
\endverse


\beginverse
Если солнце на ладони, 
Если сердце в звуках тонет,
Ты потерян для обычных дней.
Для тебя сияет полночь 
И звезда спешит на помощь,
Возвращая в дом к тебе друзей.
\endverse

\beginverse
Свой мотив у каждой птицы, 
Свой мотив у каждой песни,
Свой мотив у неба и Земли.
Пусть стирает время лица, 
Нас простая мысль утешит:
Мы услышать музыку смогли.
\endverse

\endsong


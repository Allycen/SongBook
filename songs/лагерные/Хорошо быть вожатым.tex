\beginsong{Хорошо быть вожатым}[
  by = {Даша Чамкина},
  sr = {СПО <<Алые паруса>>}]


{\nolyrics \boi $\down\q   \downx \up\q   \up \downx \up $}
\pchk

\gtab{C}{X32010}
\gtab{G}{320003}
\gtab{Am}{X02210}
\gtab{F}{133211}
\psk


\beginverse
Г\[C]де--то очень \[G]мало взрослых,
Г\[Am]де--то очень м\[G]ного вопросов,
Г\[Am]де--то между \[F]городом и лесом,
\[F]Ты найдёшь своё во\[G]жатское лето!
\bqk

Их больше, чем сорок --- вас меньше, чем два,
Мат не разрешается --- но где найти слова?
Ты за пять минут целый номер придумал
И вот тут впервые подумал\dots
\endverse


\beginchorus
\lrep\ \[C]Хорошо, хоро\[G]шо быть вожатым.
\[Am]Хорошо, хоро\[G]шо быть вожатым.
\[F]Хорошо, хоро\[G]шо быть вожатым. \rrep\ \rep 2
\endchorus


\beginverse
Ты проснулся. Ура! Ты проснулся!
Нос почесал, сам себе улыбнулся,
Новый подъём, еще одна зарядка,
Этим казявкам придётся несладко.
\bqk

Митя опять заклеил унитаз, ---
Значит, моет пол уже в который раз.
Ручки опять измазаны пастой
На каждой двери кроме вожатской!
\endverse


\beginverse
Ты читаешь сказку --- а кто такой милорд?
Ты поешь им песню --- а где найти аккорд?
Сам почти уснул и уже клюёшь носом,
Но тут ты понимаешь, что кончились вопросы.
\bqk

К стенке отвернулись, носики сопят.
Просто ангелочки (только если спят).
С дурацкой улыбкой идёшь до вожатской,
И вот тут ты понимаешь\dots
\bqk

Что опять наступаешь в лужу синей краски,
Потому что до этого в песне пелось, что
Ты им всю ночь рассказываешь сказки,
А утром наступаешь в лужу синей краски!
\endverse

\beginchorus
А ещё хорошо быть лохматым,
Хорошо, грести деньги лопатой,
Хорошо, хорошо быть бородатым,
Это потому что хорошо быть вожатым.
\bqk

Хорошо быть хорошим вожатым,
Хорошо быть весёлым вожатым,
Хорошо быть любимым вожатым,
Это потому что хорошо быть вожатым.
\endchorus

\endsong

\beginscripture{}
Песня написана Дашой Чамкиной из СПО <<Алые паруса>> в 2010 году. В том же году была исполнена на ВП ЛЭТИ в номинации <<Отрядная>>.
\endscripture

\beginsong{Изгиб гитары желтой}[
  by={Олег Митяев}]





\pchk


\gtab{Am}{X02210}
\gtab{Dm}{XX0231}
\gtab{E}{022100}
\gtab{G}{320003}
\gtab{C}{X32010}

\gtab{A7}{X02020}
\gtab{F}{133211}
\psk


\beginverse
\[Am]Изгиб \[Dm]гитары \[E]жёлтой ты обнимешь \[Am]нежно,
\[Am]Струна осколком \[Dm]эха \[G]пронзит тугую \[C]высь.
\[A7]Качнётся купол \[Dm]неба, \[G]большой и звездно-\[C]снежный\dots
\[Dm]Как здорово, \[Am]что все мы здесь \[E]сегодня \[F(Am)]собрались. \rep 2
\endverse


\beginverse
Как отблеск от заката, костер меж сосен пляшет.
Ты что грустишь, бродяга? А ну-ка, улыбнись!
И кто-то очень близкий тебе тихонько скажет:
Как здорово, что все мы здесь сегодня собрались. \rep 2
\endverse


\beginverse
И все же с болью в горле мы тех сегодня вспомним,
Чьи имена, как раны, на сердце запеклись,
Мечтами их и песнями мы каждый вдох наполним.
Как здорово, что все мы здесь сегодня собрались! \rep 2
\endverse


\beginverse
Изгиб гитары желтый ты обнимешь нежно,
Струна осколком эха пронзит тугую высь.
Качнется купол неба, большой и звездно-снежный\dots
Как здорово, что все мы здесь сегодня собрались! \rep 2
\endverse

\endsong


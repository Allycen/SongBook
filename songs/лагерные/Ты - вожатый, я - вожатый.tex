\beginsong{Ты --- вожатый, я --- вожатый}[
  by={}]

{\nolyrics \pere 5--3--2--1---3--2--3}
\ptk

{\nolyrics \boi $\down\q   \downx \up\q   \up \downx \up $}
\pchk

\gtab{Hm}{2:X13321}
\gtab{G}{320003}
\gtab{D}{XX0232}
\gtab{Em}{022000}
\gtab{F#m}{2:133111}
	
\gtab{F#}{2:133211}
\psk


\beginverse
\[Hm]Первые дни и \[F#m]первые ночи
\[G]Дети шалят, не же\[D]лая уснуть. 
Т\[Em]ри часа ночи, ус\[F#]талая вп\[Hm]рочем, 
\[Em]Ты успеваешь ко м\[F#]не заглянуть. 
\bigskip

Робко боишься в глаза посмотреть 
Припоминая свои неудачи. 
Знаешь, вожатый, надо терпеть, 
Смена пройдёт, тогда и поплачем! 
\endverse


\beginchorus
\[Em]Нет пу\[F#]ти обратно,
\[Hm]Я --- вожатый, \[G]ты --- вожатый, 
Но у костра в ночи 
Расскажи, не молчи. 
Верь, друг, ты --- вожатый, 
За тобой идут ребята. 
Но у костра в ночи не молчи. 
\endchorus


\beginverse
Утро холодное, дождик и ветер, 
Голосом хриплым объявишь <<Подъём>>. 
И, проклиная почти всё на свете, 
Крикнешь <<Отряд, на зарядку идём>>! 
\bigskip

Кто--то опять начинает болеть, 
С кем--то подрался обиженный мальчик. 
Знаешь, вожатый, надо терпеть, 
Смена пройдёт, тогда и поплачем! 
\endverse


\beginverse
Слово за делом дни полетели, 
День расставания --- значит, пора. 
Время разлуки и время потери. 
Город. Автобус. И все по домам. 
\bigskip

Не успеваешь песню допеть. 
Жаль, только смену вернуть невозможно\dots 
Знаешь, вожатый, не надо терпеть. 
Плачь, потому что сейчас это можно!
\endverse

\endsong

\beginscripture{}
Первый куплет играется перебором, далее --- боем.
\endscripture